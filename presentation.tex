\documentclass[10pt]{beamer}

\usetheme[progressbar=frametitle]{metropolis}

\usepackage{graphicx} % Allows including images

\title{RESTful APIs guidelines}
\subtitle{A Zalando showcase}
\date{\today}
\author{Lampros Papadimitriou}
\institute{Check24 Baufinanzierung GmbH}

\begin{document}

\maketitle

\begin{frame}{Principles}

  \metroset{block=fill}
  \begin{alertblock}{The Robustness Principle}
    Be liberal in what you accept, be conservative in what you send
  \end{alertblock}

  \begin{itemize}
    \item Treat you API as a product
    \item Be customer-oriented
    \item Act as a product owner
    \item Actively improve and maintain API consistency over long term
    \item Simple, comprehensive and usable API
  \end{itemize}

\end{frame}

\begin{frame}{API first!}
  \begin{columns}
   	\begin{column}{0.4\textwidth}
	  \begin{itemize}
	    \item Define API first!
		\item Follow standards
		\item Early review and feedback
		\item Free yourself from \emph{HOW} concerns
	  \end{itemize}
   	\end{column}
    \begin{column}{0.6\textwidth}
    Example
   	\begin{itemize}
   		\item Product owner (PO) receives requirement
   		\item Team designs API draft
   		\item OP communicates draft with client(s)
   		\item Review, redesign, repeat
   		\item Teams proceed with implementation
   		\item Publish API, Doc, artifact
   	\end{itemize}
   \end{column}
  \end{columns}

\end{frame}

\begin{frame}{OpenAPI}

  \begin{itemize}

    \item
          \emph{MUST} provide API specification in YAML
    \item
          \emph{SHOULD} provide API user manual online
    \item
          \emph{MUST} provide metainformation (title, description, team, audience)
    \item
          \emph{MUST} use semantic version
  \end{itemize}

\end{frame}

\begin{frame}{Security}

  \begin{itemize}

    \item
          \emph{MUST} secure endpoints with OAuth 2.0
    \item
          \emph{MUST} design and assign permissions
  \end{itemize}

\end{frame}

\begin{frame}{Versioning}

  \begin{itemize}

    \item
          \emph{MUST} never break backwards compatibility
    \item
          \emph{SHOULD} avoid versioning (MUST avoid URL versioning)
    \item
          \emph{SHOULD} prefer compatible extensions
    \item
          Clients SHOULD not crash on compatible API extensions
    \item
          MUST use media type versioning
    \item
          MUST obtain approval of clients (e.g.~deprecation)
  \end{itemize}

\end{frame}

\begin{frame}{API design}

  \begin{itemize}

    \item
          MUST always return JSON objects as top-level
    \item
          MUST property names be ASCII snake\_case (and never camelCase)
    \item
          SHOULD use additionalProperties
    \item
          SHOULD enumeration values as String
    \item
          SHOULD/MAY standards for Date, duration, currency, country, language
  \end{itemize}

\end{frame}

\begin{frame}{Useful tools}

  \begin{itemize}

    \item
          swagger online editor
    \item
          connexion
    \item
          problem / problem-spring-web
    \item
          jackson-datatype-money
    \item
          intellij-swagger
  \end{itemize}

\end{frame}

\begin{frame}{Observations about Zalando}

  \begin{itemize}

    \item
          Teams serving other teams
    \item
          Agile project structures
    \item
          Strong appearance in Github
    \item
          Techblogs
  \end{itemize}

\end{frame}

\begin{frame}{API naming}

  \begin{itemize}

    \item
          Must Use lowercase separate words with hyphens for Path Segments
    \item
          Must Use snake\_case for query parameters
    \item
          Must Pluralize resource names
    \item
          Must avoid trailing slashes
    \item
          Must stick to conventional query params
    \item
          Must avoid actions - think about resources (lvl 2)
    \item
          Must keep URLs verb-free
    \item
          Must identify resources and sub via path
    \item
          May consider not nested URLs
  \end{itemize}

\end{frame}

\begin{frame}{HTTP requests}

  \begin{itemize}

    \item
          GET for read
    \item
          PUT to update entire resources (replace)
    \item
          POST to create single resources
    \item
          PATCH update parts of single resources
    \item
          DELETE to delete resources
    \item
          HEAD to retrieve header infomation (Etag?)
    \item
          Prefer POST over PUT
    \item
          Use ETag \& If-(None)-Match
  \end{itemize}

\end{frame}

\begin{frame}{Http status codes and errors}

  \begin{itemize}

    \item
          Use standard HTTP codes
    \item
          201 CREATED (sync) vs 202 ACCEPTED (async)
    \item
          207 (multi-status)
    \item
          409 CONFLICT (concurrency problem)
    \item
          412 PRECONDITION FAILED (If-Match mismatch)
    \item
          423 Pessimistic locking
    \item
          429 Too many requests
  \end{itemize}

\end{frame}

\begin{frame}{Performance tips}

  \begin{itemize}

    \item
          SHOULD Use gzip compression
    \item
          SHOULD reduce bandwidth needs and improve responsiveness
    \item
          SHOULD support filtering fields
    \item
          SHOULD support pagination
    \item
          SHOULD allow optional embedding
    \item
          MUST Document caching, if supported (default: Cache-Control: no-cache)
  \end{itemize}

\end{frame}

\begin{frame}{Common field names}

  \begin{itemize}

    \item
          id
    \item
          xyz\_id
    \item
          created
    \item
          modified
    \item
          type
    \item
          etag
  \end{itemize}

\end{frame}

\begin{frame}{Proprietary Headers}

  \begin{itemize}

    \item
          X-Flow-ID (troubleshooting)
    \item
          X-Frontend-Type (mobile-app / browser)
    \item
          X-Device-Type (tablet, desktop)
    \item
          X-Device-OS (IOS, Android)
  \end{itemize}

\end{frame}

\begin{frame}{API operations}

  \begin{itemize}

    \item
          MUST Publish API Specifications (zalando-apis directory)
  \end{itemize}

\end{frame}

\begin{frame}{Internal processes}

  \begin{itemize}

    \item
          Use functional naming schema
    \item
          internal registry for components
  \end{itemize}

\end{frame}

\begin{frame}{Notes}

  \begin{block}{Principle}

    \begin{itemize}

      \item
            Actively improve and maintain API consistency over long term
      \item
            Simple, comprehensive and usable API
    \end{itemize}

  \end{block}

\end{frame}

\end{document}
